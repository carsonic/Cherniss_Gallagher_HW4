% Options for packages loaded elsewhere
\PassOptionsToPackage{unicode}{hyperref}
\PassOptionsToPackage{hyphens}{url}
%
\documentclass[
]{article}
\usepackage{amsmath,amssymb}
\usepackage{iftex}
\ifPDFTeX
  \usepackage[T1]{fontenc}
  \usepackage[utf8]{inputenc}
  \usepackage{textcomp} % provide euro and other symbols
\else % if luatex or xetex
  \usepackage{unicode-math} % this also loads fontspec
  \defaultfontfeatures{Scale=MatchLowercase}
  \defaultfontfeatures[\rmfamily]{Ligatures=TeX,Scale=1}
\fi
\usepackage{lmodern}
\ifPDFTeX\else
  % xetex/luatex font selection
\fi
% Use upquote if available, for straight quotes in verbatim environments
\IfFileExists{upquote.sty}{\usepackage{upquote}}{}
\IfFileExists{microtype.sty}{% use microtype if available
  \usepackage[]{microtype}
  \UseMicrotypeSet[protrusion]{basicmath} % disable protrusion for tt fonts
}{}
\makeatletter
\@ifundefined{KOMAClassName}{% if non-KOMA class
  \IfFileExists{parskip.sty}{%
    \usepackage{parskip}
  }{% else
    \setlength{\parindent}{0pt}
    \setlength{\parskip}{6pt plus 2pt minus 1pt}}
}{% if KOMA class
  \KOMAoptions{parskip=half}}
\makeatother
\usepackage{xcolor}
\usepackage[margin=1in]{geometry}
\usepackage{color}
\usepackage{fancyvrb}
\newcommand{\VerbBar}{|}
\newcommand{\VERB}{\Verb[commandchars=\\\{\}]}
\DefineVerbatimEnvironment{Highlighting}{Verbatim}{commandchars=\\\{\}}
% Add ',fontsize=\small' for more characters per line
\usepackage{framed}
\definecolor{shadecolor}{RGB}{248,248,248}
\newenvironment{Shaded}{\begin{snugshade}}{\end{snugshade}}
\newcommand{\AlertTok}[1]{\textcolor[rgb]{0.94,0.16,0.16}{#1}}
\newcommand{\AnnotationTok}[1]{\textcolor[rgb]{0.56,0.35,0.01}{\textbf{\textit{#1}}}}
\newcommand{\AttributeTok}[1]{\textcolor[rgb]{0.13,0.29,0.53}{#1}}
\newcommand{\BaseNTok}[1]{\textcolor[rgb]{0.00,0.00,0.81}{#1}}
\newcommand{\BuiltInTok}[1]{#1}
\newcommand{\CharTok}[1]{\textcolor[rgb]{0.31,0.60,0.02}{#1}}
\newcommand{\CommentTok}[1]{\textcolor[rgb]{0.56,0.35,0.01}{\textit{#1}}}
\newcommand{\CommentVarTok}[1]{\textcolor[rgb]{0.56,0.35,0.01}{\textbf{\textit{#1}}}}
\newcommand{\ConstantTok}[1]{\textcolor[rgb]{0.56,0.35,0.01}{#1}}
\newcommand{\ControlFlowTok}[1]{\textcolor[rgb]{0.13,0.29,0.53}{\textbf{#1}}}
\newcommand{\DataTypeTok}[1]{\textcolor[rgb]{0.13,0.29,0.53}{#1}}
\newcommand{\DecValTok}[1]{\textcolor[rgb]{0.00,0.00,0.81}{#1}}
\newcommand{\DocumentationTok}[1]{\textcolor[rgb]{0.56,0.35,0.01}{\textbf{\textit{#1}}}}
\newcommand{\ErrorTok}[1]{\textcolor[rgb]{0.64,0.00,0.00}{\textbf{#1}}}
\newcommand{\ExtensionTok}[1]{#1}
\newcommand{\FloatTok}[1]{\textcolor[rgb]{0.00,0.00,0.81}{#1}}
\newcommand{\FunctionTok}[1]{\textcolor[rgb]{0.13,0.29,0.53}{\textbf{#1}}}
\newcommand{\ImportTok}[1]{#1}
\newcommand{\InformationTok}[1]{\textcolor[rgb]{0.56,0.35,0.01}{\textbf{\textit{#1}}}}
\newcommand{\KeywordTok}[1]{\textcolor[rgb]{0.13,0.29,0.53}{\textbf{#1}}}
\newcommand{\NormalTok}[1]{#1}
\newcommand{\OperatorTok}[1]{\textcolor[rgb]{0.81,0.36,0.00}{\textbf{#1}}}
\newcommand{\OtherTok}[1]{\textcolor[rgb]{0.56,0.35,0.01}{#1}}
\newcommand{\PreprocessorTok}[1]{\textcolor[rgb]{0.56,0.35,0.01}{\textit{#1}}}
\newcommand{\RegionMarkerTok}[1]{#1}
\newcommand{\SpecialCharTok}[1]{\textcolor[rgb]{0.81,0.36,0.00}{\textbf{#1}}}
\newcommand{\SpecialStringTok}[1]{\textcolor[rgb]{0.31,0.60,0.02}{#1}}
\newcommand{\StringTok}[1]{\textcolor[rgb]{0.31,0.60,0.02}{#1}}
\newcommand{\VariableTok}[1]{\textcolor[rgb]{0.00,0.00,0.00}{#1}}
\newcommand{\VerbatimStringTok}[1]{\textcolor[rgb]{0.31,0.60,0.02}{#1}}
\newcommand{\WarningTok}[1]{\textcolor[rgb]{0.56,0.35,0.01}{\textbf{\textit{#1}}}}
\usepackage{graphicx}
\makeatletter
\def\maxwidth{\ifdim\Gin@nat@width>\linewidth\linewidth\else\Gin@nat@width\fi}
\def\maxheight{\ifdim\Gin@nat@height>\textheight\textheight\else\Gin@nat@height\fi}
\makeatother
% Scale images if necessary, so that they will not overflow the page
% margins by default, and it is still possible to overwrite the defaults
% using explicit options in \includegraphics[width, height, ...]{}
\setkeys{Gin}{width=\maxwidth,height=\maxheight,keepaspectratio}
% Set default figure placement to htbp
\makeatletter
\def\fps@figure{htbp}
\makeatother
\setlength{\emergencystretch}{3em} % prevent overfull lines
\providecommand{\tightlist}{%
  \setlength{\itemsep}{0pt}\setlength{\parskip}{0pt}}
\setcounter{secnumdepth}{-\maxdimen} % remove section numbering
\ifLuaTeX
  \usepackage{selnolig}  % disable illegal ligatures
\fi
\usepackage{bookmark}
\IfFileExists{xurl.sty}{\usepackage{xurl}}{} % add URL line breaks if available
\urlstyle{same}
\hypersetup{
  pdftitle={Homework: lubridate and purrr},
  pdfauthor={Carson Cherniss and Ainsley Gallagher},
  hidelinks,
  pdfcreator={LaTeX via pandoc}}

\title{Homework: lubridate and purrr}
\author{Carson Cherniss and Ainsley Gallagher}
\date{2025-02-25}

\begin{document}
\maketitle

\subsubsection{Load Packages:}\label{load-packages}

\begin{Shaded}
\begin{Highlighting}[]
\FunctionTok{library}\NormalTok{(tidyverse)}
\end{Highlighting}
\end{Shaded}

\subsection{Exercise 1: Advanced date manipulation with
lubridate}\label{exercise-1-advanced-date-manipulation-with-lubridate}

\subsubsection{Question 1: Generate a sequence of dates from January 1,
2015 to December 31, 2025, spaced by every two months. Extract the year,
quarter, and ISO week number for each
date.}\label{question-1-generate-a-sequence-of-dates-from-january-1-2015-to-december-31-2025-spaced-by-every-two-months.-extract-the-year-quarter-and-iso-week-number-for-each-date.}

\begin{Shaded}
\begin{Highlighting}[]
\CommentTok{\# Generate a sequence of dates from Jan 1, 2015, to Dec 31, 2025, every two months}
\NormalTok{dates }\OtherTok{\textless{}{-}} \FunctionTok{seq}\NormalTok{(}\FunctionTok{ymd}\NormalTok{(}\StringTok{"2015{-}01{-}01"}\NormalTok{), }\FunctionTok{ymd}\NormalTok{(}\StringTok{"2025{-}12{-}31"}\NormalTok{), }\AttributeTok{by =} \StringTok{"2 months"}\NormalTok{)}

\CommentTok{\# Create a tibble and extract year, quarter, and ISO week number}
\NormalTok{date\_tbl }\OtherTok{\textless{}{-}} \FunctionTok{tibble}\NormalTok{(}
  \AttributeTok{date =}\NormalTok{ dates,}
  \AttributeTok{year =} \FunctionTok{year}\NormalTok{(dates),}
  \AttributeTok{quarter =} \FunctionTok{quarter}\NormalTok{(dates),}
  \AttributeTok{iso\_week =} \FunctionTok{isoweek}\NormalTok{(dates) }
\NormalTok{)}

\FunctionTok{print}\NormalTok{(date\_tbl)}
\end{Highlighting}
\end{Shaded}

\begin{verbatim}
## # A tibble: 66 x 4
##    date        year quarter iso_week
##    <date>     <dbl>   <int>    <dbl>
##  1 2015-01-01  2015       1        1
##  2 2015-03-01  2015       1        9
##  3 2015-05-01  2015       2       18
##  4 2015-07-01  2015       3       27
##  5 2015-09-01  2015       3       36
##  6 2015-11-01  2015       4       44
##  7 2016-01-01  2016       1       53
##  8 2016-03-01  2016       1        9
##  9 2016-05-01  2016       2       17
## 10 2016-07-01  2016       3       26
## # i 56 more rows
\end{verbatim}

\subsection{Exercise 2: Complex Date
Arithmetic}\label{exercise-2-complex-date-arithmetic}

\subsubsection{Question 2: Given the following dates, compute the
difference in months and weeks between each consecutive
pair.}\label{question-2-given-the-following-dates-compute-the-difference-in-months-and-weeks-between-each-consecutive-pair.}

\begin{Shaded}
\begin{Highlighting}[]
\NormalTok{sample\_dates }\OtherTok{\textless{}{-}} \FunctionTok{c}\NormalTok{(}\StringTok{"2018{-}03{-}15"}\NormalTok{, }\StringTok{"2020{-}07{-}20"}\NormalTok{, }\StringTok{"2023{-}01{-}10"}\NormalTok{, }\StringTok{"2025{-}09{-}05"}\NormalTok{)}

\CommentTok{\# Convert to dates}
\NormalTok{sample\_dates }\OtherTok{\textless{}{-}} \FunctionTok{ymd}\NormalTok{(sample\_dates)}

\CommentTok{\# Calculate differences between consecutive pairs of dates}
\NormalTok{date\_diff }\OtherTok{\textless{}{-}} \FunctionTok{diff}\NormalTok{(sample\_dates)}

\CommentTok{\# Convert differences to months and weeks}
\NormalTok{months\_diff }\OtherTok{\textless{}{-}} \FunctionTok{as.period}\NormalTok{(date\_diff, }\AttributeTok{unit =} \StringTok{"month"}\NormalTok{)}
\NormalTok{weeks\_diff }\OtherTok{\textless{}{-}} \FunctionTok{as.period}\NormalTok{(date\_diff, }\AttributeTok{unit =} \StringTok{"week"}\NormalTok{)}

\NormalTok{months\_diff}
\end{Highlighting}
\end{Shaded}

\begin{verbatim}
## [1] "858d 0H 0M 0S" "904d 0H 0M 0S" "969d 0H 0M 0S"
\end{verbatim}

\begin{Shaded}
\begin{Highlighting}[]
\NormalTok{weeks\_diff}
\end{Highlighting}
\end{Shaded}

\begin{verbatim}
## [1] "858d 0H 0M 0S" "904d 0H 0M 0S" "969d 0H 0M 0S"
\end{verbatim}

\subsection{Exercise 3: Higher-Order Functions with
purrr}\label{exercise-3-higher-order-functions-with-purrr}

\subsubsection{Question 3:}\label{question-3}

\begin{Shaded}
\begin{Highlighting}[]
\CommentTok{\# R code here}
\end{Highlighting}
\end{Shaded}

\subsection{Exercise 4: Combining lubridate and
purrr}\label{exercise-4-combining-lubridate-and-purrr}

\subsubsection{Question 4:}\label{question-4}

\begin{Shaded}
\begin{Highlighting}[]
\CommentTok{\# R code here}
\end{Highlighting}
\end{Shaded}


\end{document}
